% This file is part of the i10 thesis template developed and used by the
% Media Computing Group at RWTH Aachen University.
% The current version of this template can be obtained at
% <http://www.media.informatik.rwth-aachen.de/karrer.html>.

\chapter{Introduction}
\label{introduction}

\begin{quotation}
After a long working day, Michael opens his apartment door. Or better, his door is opened for him. He has no key. Cameras at different positions in the corridor have registered his movement and his attitude. After entering his pin on a console, the DNA and the retina were scanned by another camera. An intelligent system has already approved him. As Michael enters the kitchen, his intelligent friend also sees that he is tired today. The light is dimmed, the blinds are shut down. Michael's behavior indicates that he does not want to be disturbed for the next hour. The phones and doorbell are switched to mute.

When Michael wakes up an hour later, he feels refreshed. The blinds open slightly. Michael now decides to do something for work as he often does, because he feels more relaxed there. He boots up his computer and opens the first working documents. The system is connected to the operating system of its job. It recognizes Michael's current and rapidly changing work processes. All other relevant and required information is provided discreetly while Michael starts listening to soft jazz and soul music. Michael begins to devote himself to his work.
\end{quotation}
The above scenario outlines a future working and private environment based on a networked intelligent system. It evokes reminiscences of \textit{HAL 9000}, the intelligent computer of Stanley Kubrick's Space Odysee 2001. \textit{HAL} is really intelligent, doing complex navigational computations for the spaceship and solving other computable problems. \textit{HAL} is sensitive, shows emotional reactions and bondings towards crew members, he is socialized and very human. Moreover, he is able to guess the intentions of the astronauts, knows their habits and has a strong drive for self-preservation. An emotional conflict forces him to draw his own conclusions, which lead to his deconstruction. As his neuronal circuits are slowly withdrawn from his artificial brain, \textit{HAL} suffers terribly. 

It is quite clear that intelligent machines will not be like this, but assistance and location based information systems are increasingly finding their way into everyday life. From a personal point of view, this development can be called either good or bad. Fact is, many representatives of \ac{AI} announce a quantum leap of  intelligent agents in the next decades, and more: Ray Kurzweil, Rodney Brooks, and Jeff Hawkins are of the opinion that the future of this development will be located in the so-called "'intelligent biological systems"', i.e. systems that replicate basic working principles of the brain. These will complement the previous learning-based inferential models for machine intelligence (\cite{kurzweil2013create},\cite{hawkins2007intelligence} and \cite{brooks2012brain}). 
But will these approaches be able to interpret emotions and guess about intentions? This is conceivable, if one thinks of the many sensors, which are described in the introductory example: A lower heart rate and blood pressure can indicate fatigue, facial expression can confirm this, etc. But without the physical sensors, this task gets difficult. Nevertheless, this is studied in computer science and slowly in psychology as well: How can we make sense of the "'actions"' of persons in the digital world, i.e. the websites searched, the documents read and the friends that were contacted? And moreover: Is this important at all?

Yes it is, and not just for the reason that social platforms and search engines like \textit{Facebook} and \textit{Google} gather and analyze a lot of personal data (it is always good to know, what others can do with private data). The main driving factor is the paradigm of the \textit{knowledge worker} that has become true for most of the western society and that proclaims a new working model:  The knowledge worker completes his tasks by non-routine problem-solving approaches with the help of a computer applications, the internet and social media (\cite{drucker1999knowledge}). As his work is fairly non-linear and complex, companies depend on these specialists (\cite{foss2006strategy}). In recent years the management boards of many companies became aware of this shift and established the position of a Chief Information Officers (CIO) whose work focus is information management. The objectives of this practical knowledge management go well beyond the mere supply of employes with relevant information: Employes are wanted to develop learning skills and abilities that provide added value for the company. The classification of knowledge is expressed on two poles: on the one hand the so-called explicit knowledge, which can be described and is therefore suitable to be kept in documents and on the other hand, implicit knowledge, which that can not be brought easily into tangible form. 

Getting implicit knowledge is studied in computer science under the term "'knowledge- and task mining"'. Its aim is to extract knowledge from elementary actions, i.e. operations on programs and documents, websites and social activities. 

This work tries to answer the question in what form knowledge can be extracted from interaction of users with the virtual environment. The concrete research questions are:

\fbox {
    \parbox{\linewidth}{
    \begin{enumerate}
      \item Is it possible to extract user tasks and intentions from state-of-art computer science approaches?
      \item What theoretical concepts of task analysis, goals and intentions are applied? What are the psychological foundations?
      \item If knowledge from elementary tasks can be extracted, how to handle that information?
    \end{enumerate}
    
    }
}

Questions 1. and 2. are answered in chapter \ref{relatedwork}. It will be argued that by the common approaches, so-called "'Intention-Aware Systems"' are not yet feasible. The theoretical shortcomings of the current concepts are revealed. As intentions are not extractable, there are two ways left to go. First, take existing approaches of computer science and see, how they can be integrated into organizations. Second try out a new approach and test, if the returned results are valuable and helpful on the way towards "'Intention-Aware Systems"'. The second way is handled by applying a new form of bio-inspired computing, called Hierarchical Temporal Memory (HTM), that will be explained in chapter XXX. Results will be compared to the findings of re-implemented traditional ways of knowledge-mining. 

The findings of either two approaches shall be evaluated and brought into organizational context to see, whether the extracted information is useful to a company. The evaluation and discussion are the last parts of this work.

 
