% This file is part of the i10 thesis template developed and used by the
% Media Computing Group at RWTH Aachen University.
% The current version of this template can be obtained at
% <http://www.media.informatik.rwth-aachen.de/karrer.html>.

\chapter{Introduction}
\label{introduction}

\begin{quotation}
Michael öffnet nach einem längeren Arbeitstag zuhause seine Wohnungstüre. Er hat keinen Schlüssel. Kameras an verschiedenen Positionen im Flur haben seinen Bewegungsablauf und seine Haltung registiert. Nach der Eingabe seiner Pin auf einer Konsole wurde die DNA sowie die Retina über eine weitere Kamera gescannt. Ein intelligentes System hat Michael bereits erkannt. Als er die Küche betritt erkennt sein intelligentes Apartment, dass Michael heute müde ist. Das Licht wird verdunkelt, die Jalousien werden heruntergefahren. Michaels Verhalten signalisiert dem System, dass er in der nächsten Stunde nicht gestört werden möchte. Die Telefone und Haustürklingel werden auf stumm geschalten. 

Als Michael nach einer Stunde wieder aufwacht, ist er erholt. Die Jalousien öffnen sich wieder etwas. Michael muss jetzt noch etwas für die Arbeit tun, da er oftmals besser von zuhause arbeiten kann. Er geht an seinen Rechner und öffnet die ersten Arbeitsdokumente. Das System ist gekoppelt mit dem Arbeitssystem seiner Arbeitsstelle. Es erkennt Michaels aktuellen und sich schnell ändernden Arbeitsprozess. Alle relevanten weiteren benötigten Informationen werden Michael unauffällig zur Vergügung gestellt. Michael beginnt sich seiner Arbeit zu widmen.
\end{quotation}

Das obige Zukunftsszenario skizziert eine Arbeitsumgebung und Privatumgebung, die auf einem vernetzten intelligenten System basiert. Assistenz- und ortssensitive Informationssysteme halten zunehmend Einzug in den Alltag. Diese Entwicklung kann vom persönlichen Standpunkt aus gut oder schlecht geheißen werden. Tatsache ist, dass die Entwicklung intelligenter Systeme zu einem sprunghaften Fortschritt ansetzt. Bekannte Wissenschaftler im Bereich der Artificial Intelligence wie Ray Kurzweil, Rodney Brooks und Jeff Hawkins sind der Meinung, dass die Zukunft dieser Entwicklung im Bereich der sog. "`biologischen intelligenten Systeme"' angesiedelt sein wird. Diese ergänzen die auf der Inferenzstatistik beruhenden bisherigen Lernmodelle maschineller Intelligenz um zusammen neuartige intelligente System zu bilden. The Intelligent systems that are mentioned in the example above rely on certain architecture: lots of data (so-called "Big Data") is collected via a sensoric layer. For example sensors collecting information about engergy consumption within a building or sensors that recognizes the surroundings of a building in order to detect moving persons. The accumlated raw data then has to be transferred to information and fed into a machine learning algorithm that condenses the information and is able to predict future events and deduce patterns in the flow of information. In the above example this means that that the actual energy usage is send to the computing layer. In the according model future energy values are predicted. The final part of the architecture is the feedback of the analyzed data to an output system: as in the energy example, energy peaks could be predicted and as such it is ensured that energy commming from solar heaters is supplemented by traditional energy resources. This is a typical input-computation-output approach. As machines do not depend on ears, eyes and smells, sensors can be applied to other areas as well.

 A disputed field of research is the "movement"' of persons in the digital world. The paradigm of the \textit{knowledge worker} that has become true for most of the western society proclaims a new working model:  The knowledge worker achieves his tasks by non-routine problem-solving approaches encompassing a usually non-linear sequence of steps like problem definition, information seeking, planning of solutions approach with the help of a Personcal Computer and the internet. As his work is fairly non-linear, workflows are of interest for companies to keep the knowledge as an essential good. Extracting workflows from elementary actions, i.e. operations on programs and documents, is difficult. The same documents and programs can be used in different contexts, users act in automated ways to achieve their goals, but orders of higher level activities (searching for information, elaborating presentations \dots) are permutated. This work tries to answer the question, if it is possible to extract meaningful workflows ("Process Mining") from sensoric data ("Protocol Data") by applying a new form of Biological Intelligence, called Hierarchical Temporal Memory (HTM). In analogy to the former example, the sensoric data is transmitted to the HTM algorithm, that is able not only to model behaviour but also predict next steps. The results are abstracted to information in terms of knowledge work in order to get a workflow model. The results are fed back to the user and serve as basic for a knowledge management system. 
 
 The thesis is has strong psychological implications, that will become clear in the following chapters: For one, the HTM-CLA (Cortical Learning Algorithm) was designed in analogy to the working principles of the human brain. This will the touch the areas of intention research in psychology as knowledge and knowledge acquistion are tied to intention. Second, data acquisition and analysis gained from user interaction with digital devices will become more important in the future. This work gives a hint at how this could work.
 
 The work is structured as follows: In the Related Work \ref{relatedwork} the existing approaches for knowledge mining are introduced and the problems are defined. The the HTM is elaborated and distinguished from known classical AI approaches. Its psychological relevance is emphasized and compared to the psychological research of intention. In the the chapter Own Work \ref{ownwork} the implementation of the HTM is elaborated. Experiments and results are shown. The work concludes with an outlook.
 
