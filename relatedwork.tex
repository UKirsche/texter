% This file is part of the i10 thesis template developed and used by the
% Media Computing Group at RWTH Aachen University.
% The current version of this template can be obtained at
% <http://www.media.informatik.rwth-aachen.de/karrer.html>.

\chapter{Related work}
\label{relatedwork}
This chapter introduces

\section{Digital Revolution, Knowledge Management and the Knowledge Worker}
\label{knowledgemanagement}
In the digital revolution, society experiences a shift by having computers as main tools for work and free time. The acceleration of communication influences all sectors of public life like politics, science, economy and sports\mnote{General Purpose Machine}. According to \cite{buhl1997virtuelle} the indicator for an epoch change is the all-purpose character of a Turing Machine. A computer fulfills the following functionalities:

\begin{figure}[ht]
	\centering
  \includegraphics[width=120px, height=120px]{digital}
	\caption{The All-Purpose-Machine}
	\label{fig1}
\end{figure}

The difference from a computer to a conventional machine is its faculty to applicable not only to a single domain like for example a double ram press that stamps the autobodies of a new car models. Its purpose gets clear not by the hardware but by the software and the human being that is operating the computer. Following Touring all computers are equivalent when they have to solve computable tasks. For this reason a computer has the potential to simulate all other machines (given the software for simulation is properly written). This lets \cite{buhl1997virtuelle} state the following: computers are a new entity a so-called non-machine, as they simulate processes that were formerly run with specialized machines. Second they are used as implants for specialized machines in order to fulfill a set of tasks including communication. 

As society gets more interconnected with the help of computers and the internet they form a new space - the cyberspace or virtual space. Humans communicate with machines and map information in new ways (\cite{dodge2001atlas})\mnote{Virtual Space}. Machines become portable support the workers in different situations and different contexts. But machines not only provides solutions for computable problems or communication, they also help to gather and aggregate information. The information shall become knowledge by the effort of human beings and machines that start to interpret available information. 

For organizations especially the management of knowledge is an issue (\ac{KM})\mnote{Knowledge Management}.  The goal for companies is to foster human capital and make resources available.  "'To compete effectively, firms must leverage their existing knowledge and create new knowledge that favorably positions them in their chosen markets \dots. \acs{KM} must be present in order to store, transform and transport knowledge throughout the organization"' (\cite{gold2001knowledge}). This happens in a still valid mutual agreement - as according to Taylor, the founder of Scientific Management (\cite{taylor2013scientific}, p.10):
\begin{quotation}
Scientific management \dots has for its very foundation the firm conviction that the true interests of the two (\textit{employé and employer}) are the one and the same; that prosperity for the employer cannot exist through a long term of years unless it is accompanied by prosperity for the employé, and vice versa.
\end{quotation}

According to \cite{drucker1999knowledge} the challenge for companies is to make knowledge-workers productive. Six factors determine his productivity\mnote{Knowledge Worker}:
\begin{enumerate}
  \item Productivity demands the answer to the question: \textit{What is the task?}
  \item Knowledge workers take the responsibility for their actions, they have to decide and manage themselves
  \item Continuous innovations is part of the knowledge workers' job
  \item Quantity and quality of a knowledge workers' output are equally important
  \item A Knowledge worker is an asset and  not a cost factor for the company
  \item Knowledge worker undergo a continuous process of learning and teaching
\end{enumerate}


The enlisted factors hint to the circumstance that knowledge workers do not have a clear plan or task description and many obligations: They manage multiple tasks, collaborate effectively among their colleagues and clients, and manipulate and find the information that is most relevant to them (\cite{voida2002integrating}). From a point of view of organizations, the interconnection of organizational departments and project organisation are the main characteristics of knowledge workers. But what are knowledge workers in concrete? 

According to \cite{spaeth2009info} the productivity of the production sector is always improved and rationalized whereas office routine did not take the same development. But in Germany 17 mio people work in the office. Paper as the basic medium for information has become obsolete\mnote{Knowledge Work in Organizations}. An estimated 41.2\% of German companies already digitalized their relevant information. Because of the tightly coupled interconnection between departments, knowledge work seems to be arranged on a continuum. On the low end, typcial office tasks as book keeping, finances and controlling are included in knowledge work. On the high end, scientist, managers and departmental heads need new tools and information to make their decisions. So three factors: organizational structure, IT and workplace design influence the performance of knowledge workers, but companies seem not to be able to find out how to handle the new type of work. Some companies put great emphasis on IT solutions, creating the so-called software Swiss army knife that integrates communication, collaboration, knowledge management and virtual teaming. On the other hand, many employees are overwhelmed by the amount of new tools and calculations they shall do (\cite{davenport2012mysterious}). As a last resort, emails seem to be the mainly used knowledge management tool. As such innovative tools and artificial intelligence support never succeeded, because organizational structures were not prepared to include them in their processes. It seems that the management of companies is reacting to a movement they can not control completely and there is a great need in making knowledge work more transparent. As the term "'knowledge worker"' is not new, science already suggested ways to tackle the problem of transparency and help in knowledge work. Typical approaches in information science and psychology are elaborated in the following sections.


\section{Knowledge and Task Mining}
\label{knowledgeandtask}

With the availability of more elaborate computing devices and more information, information sciences began to develop solutions to support daily work. From a historic point of view the phases can roughly be brought into a chronological order as seen in figure \ref{fig2}. This order also implies more complex tasks that rely on psychological and cognitive models. 

The first approaches with \ac{CAS} were to support users with specific information in different physical locations. With the support of natural language processing methods, the term context was widened to be usable in virtual space as well. In the virtual space a users' actions determine what is his context. In order to get information from a user's activities a psychological and explaining model is needed. At this stage the terms \ac{AAS} and \ac{IAS} were created and the field of research became more complex. In the following the scientific process is explained.


\begin{figure}[ht]
	\centering
  \includegraphics[width=120px, height=190px]{knowledge_mining}
	\caption{Knowledge and Task Mining}
	\label{fig2}
\end{figure}


In the computer scientific field of knowledge and task mining is no new subject.\mnote{Physical Context-Aware Systems} With the rise of mobile computing devices the term \ac{CAS} was created. The meaning and definition are disputed. First publications referred to a user's location: in different places usually different contextual parameters are relevant. For example a diver that is ascending from deep water has to be made aware of resting times before emerging to the surface.Another example was the \textit{Active Badge Location System} in 1992 that detected the whereabouts of a person and in order to forward relevant phone calls to telephones close targeted person (\cite{want1992active}). Such systems adapt not only to the location but also to other relevant and changing parameters in the surroundings (\cite{schilit1994context}). 

This definition was widened in 1998 where context was referred to not only the computer accessible parameters of the surroundings but also the emotional state, focus of attention, date and time as well as people in the environment (\cite{dey1998context}). The new aspect of internal parameters like focus of attention was then referred to a further elaboration of the definition\mnote{Internal Context Parameters}: the internal (logical) and external context: Internal context parameters are specified by the user in interaction with the computer like goals, tasks, work context, business processes and emotional state. External parameters are usually measured by hardware sensors, i.e. location, light, sound, movement, temperature, pressure etc. (\cite{hofer2003context}). The contextual parameters can be grouped into four categories: identity (marked by a unique identifier), the location (an entity’s position), activity (status, meaning the intrinsic properties of an entity, e.g., temperature and lightning for a room, processes running currently on a device etc.) and time (timestamps, \cite{dey2001conceptual}). 

An example of the use of internal data for extracting context is the \textit{Watson Project} \cite{budzik2000user}. Here the focus is shifted for collecting contextual information from user interaction with the computer in order to proactively support the user\mnote{Proactive User Support}. Proactivity is a term that  originates in organizational psychology and describes the ability of workers to not react to situations, but sense upcoming situational changes in advance and take control (\cite{grant2008dynamics}). As work gets more dependent of the retrieval and analysis of information, a proactive support system shall help the user in his various tasks by providing him with relevant information. 

This approach had further implications as gathering information from the user interaction with his computer requires techniques from information retrieval and computer linguistics\mnote{Natural Language Processing and Context Based Recommender Systems}. In this case the documents a user works with are analyzed and keywords are stored as vectors or a bag of words. The relevant keywords shall help to narrow the topical context a user is working on. Keywords than help to start searches with relevant search terms and provide the user with the information he needs (\cite{budzik2000user}). As a single user is often not able to find the needed information, his typical search patterns are compared with those of other users. In these cases as \textit{user model} is created, and his search terms are compared to those of other users' and the documents they found. If keywords are matching, documents of those others users are recommended (\cite{anand2007contextual}). This approach is called \ac{CBRS} recommendation and their related techniques like user-collaborative filtering are applied in search engines like Amazon \footnote{www.amazon.com}.

\ac{AAS} at last have different focus: The guiding principle of \acs{AAS} is that users have limited cognitive resources and are distracted easily \mnote{Attention-aware systems}. They suffer from an \textit{information overload} as they jump quickly from one resource to the next in the same and different workings tasks. Whilst it is beneficial to be able to change foci in certain situations, in others it is exhausting. Therefore systems capable of supporting and guiding user attention have to assess the current user focus, and calculate the cost/benefits of attention shifts (interruptions). As this explanation shows, \acs{AAS} have a foundation in cognitive psychology, i.e. how attention is elicited, distracted and shifting over time. Experimental setups include multiple sensor arrays like  gaze-tracking-, gesture-tracking, speech-detection and systems that measure the physiological cues (\cite{roda2006attention}). But there are also non-sensory based approaches that record users' interaction with software (\cite{horvitz2003models}, \cite{schmitz2011contextualized}). Attention management architectures expand the agenda of context-based systems, as they want not only to detect the current state of the attention of users, but also want provide support. Therefore not only the attentional state has to be tracked but the system needs to establish the users' goals and current tasks and also the happenings in the environment (\cite{roda2006attention}). 

Consequently this lead to the proclamation \ac{IAS}. This approach combines \acs{CAS} and \acs{AAS} by explicating individual and implicit intentions and plans of users' to reason about attention and context information. Dealing with context and attention means dealing with uncertainty \mnote{Intention-Aware Systems}. Explicated task models, so the idea, could help to increase the chances in proactive support. The term "'intention"' is  approached in the following way(\cite{cohen1990intention}):
\begin{quotation}
  Intention has often been analyzed differently from other mental states such as belief and knowledge\mnote{What is Intention?}. First, whereas the content of beliefs and knowledge is usually considered to be in the form of propositions, the content of an intention is typically regarded as an action. For example, Castefiada treats the content of an intention as a "`practition"' similar to an action description \dots. It is claimed that by doing so, and by strictly separating the logic of propositions from the logic of practitions, one avoids undesirable properties in the logic of intention, such as the fact that if one intends to do an action a one must also intend to do a or b. However, it has also been argued that needed connections between propositions and practitions may not be derivable.
\end{quotation}

The authors further argue that intention is directed towards the future actions and according plans. Intention thus shall be modeled as "'a composite concept of what an agent has chosen and how the agent is committed to that choice"'. The choice can be a desire or goal. Intention therefore can be described as a persisting goal. If intention is defined in a formal theory, then beliefs, goals and desires must be expressed in the same way. As the theory may be correct, the deductions fall short for real world problems. On the other hand, if those terms are used in a very abstract way, they can not be used for a touring machine. 

The following approach is an example In \cite{schmidt2011task} intention is externalized in task models\mnote{Modeling User and Intention}. 

\begin{figure}[ht]
	\centering
  \includegraphics[width=\textwidth, height=120px]{k_model}
	\caption{K-Model}
	\label{fig1}
\end{figure}

The basis for this approach is a simple cognitive Human-Interaction-Model (\textit{K-System-Model}) as shown in figure \ref{fig1}. The human being is composed of a perceptor, operator, and an effector. The components: attention, planning and intention are seen as motivator according the definition of intention mentioned above. The environment is seen as context divided into three components: things directly related to human intention (intrinsic context), unrelated external context and things that are not perceived. Context-aware and attention-aware systems are included in this model if user attention can be guided: i) intrinsic context features are provided in a user-friendly manner, ii) deficits of selections of intrinsic and extrinsic context features are corrected by shifting irrelevant features to the extrinsic context and vice versa and iii) unperceived things a are brought to user awareness. 

This first model does not answer the question how intention can be operationalized. Therefore they introduce the term \textit{task} and \textit{task models}\mnote{Task and Task Models}. If task objectives are described including further information about task execution processes they lead to a plan that operationalizes intentions.  In this line of thought, humans are seens as operands and their behavior is analyzed according to a clear set of measures. This mechanic like definition is also visible in the model above (figure \ref{fig1}). Existing task models in \ac{ICT} apply different modeling methods but the same approach towards the analysis of behavioural traces. It is obvious, that the behavioral patterns must in some way be connected to tasks or goals. The way to do this is by a. the means of describing the tasks, b. the methods for culstering the behavioral traces and connecting them to the tasks. In general there are two approaches to describe tasks: a. model the tasks and goals in advance. This can be achieved by describing tasks hierarchically (\cite{newell1972human}), or as a sequence of actions with a defined (\cite{eder1995workflow}) order. If actions and tasks are not described in advance, they usually do not have a pre-defined order or structure. 

In this case, machine-learning technologies are used to extract regularities that can be named as tasks (\cite{schmitz2011contextualized})\mnote{Task Extraction and Machine Learning}. The second approach is eligible as the modeling of tasks is usually a very tedious assignment and then well-defined description do not match working processes in the real world. If task or coherent sequences of actions are found and named, the next job is to cluster them according to so-called activity schemes, that match the higher level descriptions of intentions as typical tasks of knowledge workers: Analise, acquire, disseminate, search and communicate information. With this again, typical classification of knowledge workers' roles shall be made possible: Learners, linkers, networkers etc. can be identified (\cite{reinhardt2011knowledge}). The machine-learning approaches will be explained in more detail in the next chapter. 

\section{Psychological Theories}
As was found out in chapter \ref{knowledgeandtask}, \acs{ICT} was forced to find methods to group and define fuzzy terms like intention and tasks. In the following the theoretical approaches behind the concepts shall be elaborated. The first models tended to be observational, analyzing tasks in normal jobs with given goals (Scientific Management). This model was transmitted to the field of \acs{ICT}(MHP and GOMS). The results there are to fine grained and constrained as they need explicit task models and process description. A need for a broader model was pronounced. This was achieved with the approaches of \ac{AT}, \ac{ANT} and \ac{DC}. These models are very fruitful in \ac{HCI} but treat the results as artifacts, i.e. tools used and tasks discovered are useful when they are fed back to the users and the organization. They are not helpful in discovering intentions or tasks from user observation. The this end a psychological approach is proposed that connects habits and intentions. With this it will be possible to operationalize tasks and intentions and give meaning to the terms.

\begin{figure}[ht]
	\centering
  \includegraphics[width=200px, height=120px]{psychology}
	\caption{Psychological Approaches}
	\label{fig2}
\end{figure}


\subsection{Scientific Management}
Task analysis is no new invention: famous task analysis were done by Taylor (Scientific Management) and Gilbreth (\cite{taylor2013scientific}, \cite{gilbreth1911motion}). The approach was connected to the new ways of industrialization and assembly lines. Their goal was to analyize working tasks in order to find solutions that are efficient and not exhausting for the worker, analyzing every single working step for optimization. Gilbreth outlined the steps in analyzing a task as follows: 1. Reduce practice to writing (i.e. stop work and write down). 2. Enumerate motions used. 3. Enumerate variables which affect each motion. Three categories of variables were considered in a motion study: characteristics of the worker (e.g., physical build, experience, temperament), characteristics of the surroundings (e.g., lighting, tools), and characteristics of the motion (e.g., direction, length, speed) \cite{creighton1992origin}.

\subsection{\ac{MHP} und \ac{GOMS}}
The \ac{MHP} is a cognitive approach that helps to determine the time needed to complete a task in \ac{HCI}. This is in resemblance to the approaches described above by Taylor and Gilbreth but with basics from cognitive psychology and psychophysics: the human is seen as an information processing system in accordance with a machine. The different parts of that machine, the Perceptual Processor, Working Memory with Visual and Auditory Image Store, a Cognitive Processor and Motor Processor as well as a Long-Term Memory have a Storage Capacity $\mu$, a Decay Constant $\delta$, a Cycle Time $\tau$ and a Main Code Type $\kappa$. The time needed to fulfill a task is constrained by the speed of the separate faculties. An example is the experiment of drawing a line back and forth between two parallel lines. The motor processor can issue commands about once every $\tau=70$ msec. This leads to a certain number of pen reversals within a defined time period. The perceptual system can see whether the strokes are correctly drawn between the two lines. The percption has $\tau=100$ msec and sends the information to the cognitive system with a decision time of $\tau=70$ msec. The correction then takes again $\tau=70$ msec. Total correction time therefor is $240$ msec. As a conclusion, if a test person draws as rapidly as he can, corrections occur at a different frequency as the simple drawing of lines (\cite{card1986model}). It can be seen that this way of measuring task time needs a well-defined task description as proposed by \cite{annett1967task} and even on a granular level. The same holds true for the extension of the \ac{MHP} to \ac{HCI}: \ac{GOMS} is a framework the maps the different steps in \ac{MHP} to the processes in \ac{HCI}. \ac{GOMS} assumes that routine cognitive skills can be described as a serial sequence of of cognitive operations and motor activities within the a computer session (\cite{olson1990growth}):



\begin{figure}[ht]
	\centering
  \includegraphics[width=320px, height=280px]{goms}
	\caption{GOMS}
	\label{fig3}
\end{figure}

An example of a typical study is the following: a user has several ways of entering digits into a spreadsheet application: with a mouse or by using the keyboard. The mouse method took $4.19$ sec in average, the keyboard $2.46$ sec. The mouse method was calculated in the following way:

\begin{tabbing}
das ist nur ein saudaemlicher test fur die spalten\=  hier die zeit\kill
Moving the hand to the mouse \> 360 msec\\
Clicking the mouse \> 230 msec\\
Moving the hand to the keyboard \> 360 msec\\
Retrieving two digits \> 1200 msec\\
Typing two digits (each)\> 460 msec\\
Retrieving the end action \> 1200 msec\\
Typing the <ret> key \> 230 msec\\
\_Total \> 4040 msec
\end{tabbing}

As \ac{GOMS} is very exact for a well-defined task, its usefulness declines when a task is not clearly described, when there are too many choices for users, too much parallel work and cognitive load. 

\subsection{Activity Theory}
Besides the behavioristic approaches mentioned in chapter \ref{knowledgeandtask}, task analysis had another impact in the field of \ac{HCI} in the form of the \ac{AT}. \acs{AT} is a psychological metatheory developed in Russia with its main protagonists being Vygotsky, Rubinshtein, Leont'ev, Zeigarnik and Ovsiankina  and in its original ideas inspired by Lewin. Activity theorists, although developing a meta theoretic terminology, were interested in solving practical problems like helping mentally or physically handicapped children, educational testing and ergonomics etc. Activity theory is a powerful and clarifying descriptive tool rather than a strongly predictive theory (\cite{nardi1996activity}). Activity theory begins with a general criticism of the subject and interpretation in psychology (\cite{leont1974problem}): Psychology separates object and subject in order to get a direct relation in the form of, f.ex. $S-R$ approaches, whether they are cognitively mediated or not. Another example is a typical experimental setup that artificially creates an environment (a so-called \textit{standardized} environment as a \textit{ceterus paribus} condition), that does not fit the socio-economical surroundings of the human being. Nevertheless it generalizes its findings to that extent. Like in cybernetics it seems to postulate a cyclic feedback between an actor and its surroundings: 

\begin{figure}[ht]
	\centering
  \includegraphics[width=100px, height=40px]{sor}
	\caption{Feedback of SOR}
	\label{fig2}
\end{figure}

But the feedback mechanism is more complex, as the \textit{persona} becomes visible in its activity that is influenced, motivated and guided by cultural and internalized artifacts. The person thus is acting upon the world and changing it, changing culture and thus changing himself again. To understand its motivation, external activity must be observed and brought in relation with the other factors. Activity therefore theorists argue that consciousness is not a set of discrete isolated cognitive acts (decision making, classification, remembering, reasoning), and certainly it is not the brain: consciousness is located in everyday practice. Doing is firmly embedded in the social matrix of which every person is an organic part. The social matrix is composed of people and artifacts. Artifacts may be physical tools or sign systems such as human language. Understanding the interaction of the individual, other people, and artifacts in everyday activity is the challenge activity theory has set for itself (\cite{nardi1996activity}). The complex arrangement can be seen in figure \ref{fig3} (\cite{bryant2005becoming}).

\begin{figure}[ht]
	\centering
  \includegraphics[width=220px, height=180px]{activity_theory}
	\caption{Components of Activity Theory}
	\label{fig3}
\end{figure}


As this approach is complex it is strongly simplified in task analysis. As f.ex. described in an experiment by Rattenburry (\cite{rattenbury2007caad}) using an unsupervised system called CAAD to detect tasks:
\begin{quotation}
We draw primarily from Activity Theory (AT). Activities are the key structure in AT. They are composed of a subject, tools and an objective. The subject is the person, or persons, motivated to carry out and achieve the objective of the activity. The actions performed in an activity are mediated by tools. Tools include everything from found objects like sticks to manufactured objects like hammers to abstract, non-physical objects like words and ideas. In terms of CAAD, users are subjects and documents, folders, applications, and email addresses are tools. In the next section, we discuss how CAAD finds, represents, and uses context structures. Activities are generally long-term structures whose stability derives from their motivating objective. In working on an activity, however, people tend to focus on shorter-term goals. These goals organize the actions that people perform e.g. sending an email, writing a section of a paper, or painting a room. Both actions and the activities they service involve a fairly stable set of subjects (i.e. people) and tools. This stable set of people and tools constitutes the context structure of the user’s action and activity. CAAD searches for these stable sets in the event logs it gathers.
\end{quotation}

\subsection{Actor Network Theory and Distributed Cognition}
The \ac{ANT} is related to the \acs{AT} as it takes a similar analytical position: Human and their functionality can not be described independently of the tools they are working with. For example a scientist would not be a scientist anymore if he was deprived of his desk, his journals, books and computer. The scientist in his function as a scientist is working and interacting with a \textit{heterogenous network} (\cite{law1992notes}). This network and the interactions are the basic building block for understanding the organization as whole, may it be a company, a state or another kind of union. In this view, machines and humans are not separated, they are part of a bigger system. Humans usually don't interact without tools, may this be a blackboard or a beamer. Interaction is  \textit{mediated}.
\begin{quotation}
  \dots what counts as a person is an effect generated by a network of heterogenous, interaction materials. \dots people are who they are because they are a patterned network of heterogeneous materials. \dots So when \acs{ANT} explores the character of an organization, it treats this as an effect or a consequence - the effect of interaction between materials and strategies of the organization (\cite{law1992notes}).
\end{quotation}

The theory of \ac{DC} uses the \acs{ANT} as a foundation and elaborates it to be usable in the sens of the \acs{AT}. At first the term \textit{cognition} is pushed forward again. It can be thought of as the inner representation of a \textit{heterogeneous} interaction (\cite{hutchins2000distributed}). A process here is not cognitive because it happens in the brain. It is enclosed in the relationship among the elements that participate. The guidelines are (\cite{hollan2000distributed}):
\begin{enumerate}
  \item Cognitive processes are distributed across the members of a social group, including emerging phenomena of social interactions
  \item Cognitive processes involve coordination between internal and external (material or environmental) structure
  \item Cognitive processes are mediated by culture
\end{enumerate}

As the cognitive processes are brought to light by activities the main focus of observation are \textit{events}. But events are not isolated or a mere collections of observational data, they have to be brought into the context of the situation and require different observational technologies (Interviews, Audio, Video). They show how information is arranged by interaction. The complex arrangement of the \ac{DC} approach can be seen in figure

\begin{figure}[ht]
	\centering
  \includegraphics[width=200px, height=100px]{distributed}
	\caption{Research areas in \ac{DC}}
	\label{fig4}
\end{figure}

Figure \ref{fig4} resembles figure \ref{fig2}. But \ac{DC} extends the approach of \ac{AT} as it specifies how experiments should be done: taking the aspects of figure \ref{fig4} into account the experiments reveals the interactions between the components. As interaction includes the experiment itself, or the new tools, it becomes an artifact in itself. Experiments are seen as "'settings in which people make use of variety of material ans social resources in order to produce socially acceptable behavior."'. Experiments, if promising, are re-run in order to see changes in the distributed cognition. This iterative approach fits the change in the workflow as more organic. A typical example for a distributed designed experiment is a study done by \cite{denef2008handy}, where an orientation system in burning and smoking buildings for fire fighters was developed. The solutions proposed were tested in real world situations with fire fighters and strongly discussed afterwards. The aim was not only to find a technically performing platform but a new solutions that fit the operation plans of firefighters and even enhance their procedures in natural settings.

\subsection{Habits and Intention}
\label{habits}
Habits play an important role in psychology, although research is not so well established\mnote{Habits}. Early sociologists recognized the useful concept for social institutions(\cite{weber1946social}, \cite{mead2007movements}, \cite{durkheim1933division}). William James suggested that habits have motivational properties, provide continuity to experience and behavior and uphold social structures (\cite{james2011principles}). Of course, habits had an outstanding role in the psychological movement called 'behaviorism' in theories of Watson, Skinner and Hull\cite{ouellette1998habit}. In modern information-processing approaches habits are seen as automated responses to a stable context with high opportunity, emerging from response repetitions (\cite{ronis1989attitudes}). As they are automated, the are usually executed subconsciously, leaving cognitive processing power for superordinate tasks. 

Intentions are defined in accordance with chapter \label{knowledge and task}\mnote{Intentions}. According to Heckhausen, intentions resemble plans about how to act when predetermined conditions occur. Once formed, they gradually become automatic operations or habits (\cite{heckhausen1990intentional}). Intentions are also motivating because they become salient when the outcomes of an act can be predicted. They thus form attitudes towards behaviour (\cite{ouellette1998habit})

Why are habits important?\mnote{Habits and Intentions Integrated} Because the help to predict future behavior. Or as outlined by Triandis: The frequency of past behavior is a standard indicator for habit strength (\cite{triandis1979values}). And why are intentions important? Because they are also formed by attitudes, believes and habits, as habits are integrated into the self-concept (\cite{ouellette1998habit}, \cite{festinger1962theory}, \cite{bem1973self}). First, habits differ from intentions and attitudes in scope. Intentions reach from the general to the specific, habits are always specific and limited in scope (\cite{allport1935attitudes}). Second, intentions play an important role in learning, when situations are encountered for the first time. Here intentions include details of how to handle the situations and the different aspects. They are formed and executed consciously. An example is a little child that learns to use the telephone, holding the receiver close to the ear, then listen and talk. With the practice in the constant context, intentions are represented in a broader and more efficient manner. They reflect more stable and long term strategies like: calling a friend to organize a party etc (\cite{heckhausen1990intentional}). So intentions are chunked and packed into higher units that have automatic but as well controlled and conscious components. The typical flow of semiautomatic response patterns is a string of autonomous phases, where each phase is completed before the next will start. Between the phases, control acts are necessary, either to start the next process or to stop the flow of actions (\cite{bargh1989conditional}). 

\begin{figure}[ht]
	\centering
  \includegraphics[width=350px, height=250px]{intentions1}
	\caption{Action Flow with Intentions}
	\label{fig5}
\end{figure}

An example for an action flow with intentions is figure \ref{fig5}\mnote{Intention Action Plan}: The main intentions is to develop an administration tool for libraries. The next intention is to collect the necessary requirements. One important source for this is to call the costumer and see what he needs. This step for requirements engineering is executed. The example also clarifies a typical problem when researching intentions: if an intentional plan is executed, steps and their order can vary. For example the telephone call can be done after the phone number is found on the internet, rather than by searching older emails etc. Or emails are searched in another context, i.e. for another project. This leads to the conclusion that without further semantic information, simple actions can not be grouped to higher units of intentions as intentions are not only formed by former behavior but also by world knowledge (\cite{baldwin2001discerning}, \cite{bandura1977social}). So the question of interest is how intentions are acquired by children:

\cite{baldwin2001discerning} argue that for understanding and chunking elementary actions to higher units, an understanding of the intentions of action performers is necessary. In the ongoing flow of actions, intentional cues give meaning to stream of activity. Cues can be physical and emotional, as for example a mother, explaining a child that it is dangerous to touch a hot plate by pointing to the plate, moving the hand over the plate and expressing signs of pain. This pattern of sensing environmental objects, moving towards them or showing other signs of recognition, are key patterns in detecting intentional acts. 
In accordance with this observation, children and grown-ups seem to have the ability of detecting statistical patterns in elementary actions: Siskind and Thibadeau argue, that these actions can be computationally modeled and used to make computers detect intentions (\cite{thibadeau1986artificial}, \cite{siskind1995grounding}). 

\section{Machine Learning and Knowledge Management}
Most KM and \ac{DM} techniques involve learning patterns from existing data or information, and are therefore built upon the foundation of machine learning and artificial intelligence. The primary techniques that can be used by the organizations usually are statistical analysis, pattern discovery and outcome prediction. A variety of non-typical data can be similarly monitored. Before the advent of \acs{DM} and \acs{KM} techniques, the organizations relied almost exclusively on human expertise (\cite{tsai2012knowledge}). In the following the general approaches in \acs{DM} for \acs{KM} are discussed.

\subsection{Supervised Machine Learning}
A typical example for supervised task learning is an approach called 'bag of words'-modeling. In the 'bag of words'-model a text is represented as an unordered collection of words, where the frequency of each word is used as a feature for training a classifier. \cite{granitzer2008analysis} use this basic method with the \ac{TF-IDF} measure from the field of \ac{IR} as the input for their classifiers which is the final step of their processing pipeline. On the first level, the so-called \textit{data acquisition}, raw event data from the operating system is collected. These are keystrokes, mouse clicks and used applications as well as file names, file authors, document structure etc. User actions and operating system reactions are called \textit{events} (see fig. \ref{fig2}).  
\begin{figure}[ht]
	\centering
  \includegraphics[width=240px]{granitzer}
	\caption{Classifying events}
	\label{fig2}
\end{figure}

In this case, subsequent events are aggregated to so-called \textit{event blocks}. The rule for creating these blocks can be 'time': events that take place within a small time period, or semantic characteristics defined by applications like \textit{editing} a text file. An example for this mapping would be: A user opens a text file with his text processing application, navigates to a certain paragraph, begins reading and then writing (as reading is usually recognized by scrolling within the application). These would map to an event block with a corresponding semantic meaning (see below) that could be called: \textit{edit a word document}. Event blocks have \textit{features or attributes} that are part of the event log format. The features used by the authors were: Application name, window title, content and semantic type. Of these, the semantic type is the prevalent feature described above for building event blocks, if the application provides according detailed information. If this is not the case, data is furthermore preprocessed to be used as a 'bag-of-words' for each event block. To this ends the features are summarized in word vector, stopwords are removed and then the words are \textit{stemmed} which means the words are reduced to their root. An example for this is the stem 'dog' that is extracted from words like: doggy, doglike, dogs etc. To get the meaningful terms the \acs{TF-IDF}-measure is computed, that extracts meaningful words from the event-block. The result then is used as a classifier for the machine-learning algorithms.  Classifying hereby is done with a supervised approach where users train the algorithms with task labels given to the found event-block clusters. Classifiers used were \ac{KNN}, \ac{NB} and \ac{SVM}. The authors report an accuracy rate for this approach with an average of $Â = 74.1$ with a standard-deviation of $\sigma = 8.2$.

\subsection{Unsupervised Machine Learning}
Unsupervised approaches also use clustering algorithms to group contextual artifacts. In \cite{rattenbury2007caad} the man-machine interaction was logged according to a pull approach: this means not operating system was triggering the events but a special program that checked for events every 2 seconds. 

\begin{figure}[ht]
	\centering
  \includegraphics[width=200px,height=130px]{caad}
	\caption{Classifying events}
	\label{fig3}
\end{figure}

Figure \ref{fig3} shows the timeline of activated applications (firefox, winword, gimp, excel, explorer).

\begin{center}
  \begin{math}
  D=
  \bordermatrix{
  & 1 & 2 & 3 & 4 & 5 \cr
  & 15 & 10 & 5 & 10 & 5\cr
  & 15 & 10 & 15 & 15 & 15}
  \end{math}\\
\end{center}

If the logger checks for activated applications every to seconds, a frequency matrix composed of 30 second time intervals looks like the matrix depicted above. The first row marks the application numbers. The subsequent rows for each column show the activation frequency in the 30 second time interval if an activation check occurs every two seconds. \textit{Firefox} was active for the whole shown 60 seconds thus the first column of the matrix has to entries with the value $15$. \textit{Winword} on the other hand, was active from second 10 to second 50 resulting in two entries of the value $10$ etc. A single row shows the context structure with in the 60 second time period. The values show the probabilities of observing a certain artifact within the context. The non-negative matrices are then feed into an algorithm called \ac{GaP} being a subform of \ac{LSA} (\cite{canny2004gap}). As a result users get view of their task-related context-features in unobtrusive cloud tags.

\begin{figure}[ht]
	\centering
  \includegraphics[width=130px,height=130px]{cloud}
	\caption{Context cloud}
	\label{fig4}
\end{figure}

  
