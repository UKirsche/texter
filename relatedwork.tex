% This file is part of the i10 thesis template developed and used by the
% Media Computing Group at RWTH Aachen University.
% The current version of this template can be obtained at
% <http://www.media.informatik.rwth-aachen.de/karrer.html>.

\chapter{Related work}
\label{relatedwork}
\section{Knowledge and Task Mining}
In the computer scientific field of knowledge and task mining is no new subject.\mnote{Context-Aware Systems} With the rise of mobile computing devices the term \ac{CAS} was created. The meaning and definition are disputed. First publications referred to a user's location: in different places usually different contextual parameters are relevant. For example a diver that is ascending from deep water has to be made aware of resting times before emerging to the surface.Another example was the \textit{Active Badge Location System} in 1992 that detected the whereabouts of a person and in order to forward relevant phone calls to telephones close targeted person (\cite{want1992active}). Such systems adapt not only to the location but also to other relevant and changing parameters in the surroundings (\cite{schilit1994context}). This definition was widened in 1998 where context was referred to not only the computer accessible parameters of the surroundings but also the emotional state, focus of attention, date and time as well as people in the environment (\cite{dey1998context}). The new aspect of internal parameters like focus of attention was then referred to a further elaboration of the definition: the internal (logical) and external context: Internal context parameters are specified by the user in interaction with the computer like goals, tasks, work context, business processes and emotional state. External parameters are usually measured by hardware sensors, i.e. location, light, sound, movement, temperature, pressure etc. (\cite{hofer2003context}). The contextual parameters can be grouped into four categories: identity (marked by a unique identifier), the location (an entity’s position), activity (status, meaning the intrinsic properties of an entity, e.g., temperature and lightning for a room, processes running currently on a device etc.) and time (timestamps, \cite{dey2001conceptual}). An example of the use of internal data for extracting context is the \textit{Watson Project} \cite{budzik2000user}. Here the focus is shifted for collecting contextual information from user interaction with the computer in order to proactively support the user. Proactivity is a term that  originates in organizational psychology and describes the ability of workers to not react to situations, but sense upcoming situational changes in advance and take control (\cite{grant2008dynamics}). As work gets more dependent of the retrieval and analysis of information, a proactive support system shall help the user in his various tasks by providing him with relevant information. This approach had further implications as gathering information from the user interaction with his computer requires techniques from information retrieval and computer linguistics. In this case the documents a user works with are analyzed and keywords are stored as vectors or a bag of words. The relevant keywords shall help to narrow the topical context a user is working on. Keywords than help to start searches with relevant search terms and provide the user with the information he needs (\cite{budzik2000user}). \mnote{Context-based Recommender Systems} As a single user is often not able to find the needed information, his typical search patterns are compared with those of other users. In these cases as \textit{user model} is created, and his search terms are compared to those of other users' and the documents they found. If keywords are matching, documents of those others users are recommended (\cite{anand2007contextual}). This approach is called \ac{CBRS} recommendation and their related techniques like user-collaborative filtering are applied in search engines like Amazon \footnote{www.amazon.com}.\ac{AAS} at last have different focus: The guiding principle of \acs{AAS} is that users have limited cognitive resources and are distracted easily \mnote{Attention-aware systems}. They suffer from an \textit{information overload} as they jump quickly from one resource to the next in the same and different workings tasks. Whilst it is beneficial to be able to change foci in certain situations, in others it is exhausting. Therefore systems capable of supporting and guiding user attention have to assess the current user focus, and calculate the cost/benefits of attention shifts (interruptions). As this explanation shows, \acs{AAS} have a foundation in cognitive psychology, i.e. how attention is elicited, distracted and shifting over time. Experimental setups include multiple sensor arrays like  gaze-tracking-, gesture-tracking, speech-detection and systems that measure the physiological cues (\cite{roda2006attention}). But there are also non-sensory based approaches that record users' interaction with software (\cite{horvitz2003models}, \cite{schmitz2011contextualized}). Attention management architectures expand the agenda of context-based systems, as they want not only to detect the current state of the attention of users, but also want provide support. Therefore not only the attentional state has to be tracked but the system needs to establish the users' goals and current tasks and also the happenings in the environment (\cite{roda2006attention}). Consequently this lead to the proclamation \ac{IAS}. This approach combines \acs{CAS} and \acs{AAS} by explicating individual and implicit intentions and plans of users' to reason about attention and context information. Dealing with context and attention means dealing with uncertainty \mnote{Intention-Aware Systems}. Explicated task models, so the idea, could help to increase the chances in proactive support. The term "'intention"' is  approached in the following way(\cite{cohen1990intention}):
\begin{quotation}
  Intention has often been analyzed differently from other mental states such as belief and knowledge. First, whereas the content of beliefs and knowledge is usually considered to be in the form of propositions, the content of an intention is typically regarded as an action. For example, Castefiada treats the content of an intention as a "`practition"' similar to an action description \dots. It is claimed that by doing so, and by strictly separating the logic of propositions from the logic of practitions, one avoids undesirable properties in the logic of intention, such as the fact that if one intends to do an action a one must also intend to do a or b. However, it has also been argued that needed connections between propositions and practitions may not be derivable.
\end{quotation}

The authors further argue that intention is directed towards the future actions and according plans. Intention thus shall be modeled as "'a composite concept of what an agent has chosen and how the agent is committed to that choice"'. The choice can be a desire or goal. Intention therefore can be described as a persisting goal. If intention is defined in a formal theory, then beliefs, goals and desires must be expressed in the same way. As the theory may be correct, the deductions fall short for real world problems. On the other hand, if those terms are used in a very abstract way, they can not be used for a touring machine. The following approach is an example In \cite{schmidt2011task} intention is externalized in task models. 

\begin{figure}[ht]
	\centering
  \includegraphics[width=\textwidth, height=120px]{k_model}
	\caption{K-Model}
	\label{fig1}
\end{figure}

The basis for this approach is a simple cognitive Human-Interaction-Model (\textit{K-System-Model}) as shown in figure \ref{fig1}. The human being is composed of a perceptor, operator, and an effector. The components: attention, planning and intention are seen as motivator according the definition of intention mentioned above. The environment is seen as context divided into three components: things directly related to human intention (intrinsic context), unrelated external context and things that are not perceived. Context-aware and attention-aware systems are included in this model if user attention can be guided: i) intrinsic context features are provided in a user-friendly manner, ii) deficits of selections of intrinsic and extrinsic context features are corrected by shifting irrelevant features to the extrinsic context and vice versa and iii) unperceived things a are brought to user awareness. This first model does not answer the question how intention can be operationalized. Therefore they introduce the term \textit{task} and \textit{task models}. If task objectives are described including further information about task execution processes they lead to a plan that operationalizes intentions. Task analysis is no new invention: famous task analysis were done by Taylor (Scientific Management) and Gilbreth (\cite{taylor2013scientific}, \cite{gilbreth1911motion}). The approach was connected to the new ways of industrialization and assembly lines. Their goal was to analyize working tasks in order to find solutions that are performant and not exhausting for the worker, analyzing every single working step for optimization. Gilbreth outlined the steps in analyzing a task as follows: 1. Reduce practice to writing (i.e. stop work and write down). 2. Enumerate motions used. 3. Enumerate variables which affect each motion. Three categories of variables were considered in a motion study: characteristics of the worker (e.g., physical build, experience, temperament), characteristics of the surroundings (e.g., lighting, tools), and characteristics of the motion (e.g., direction, length, speed) \cite{creighton1992origin}. In this line of thought, humans are seens as operands and their behaviour is analyzed according to a clear set of measures. This mechanic like definition is also visible in the model above (figure \ref{fig1}). Existing task models in \ac{ICT} apply different modeling methods but the same approach towards the analysis of behavioural traces. It is obvious, that the behavioural patterns must in some way be connected to tasks or goals. The way to do this is by a. the means of describing the tasks, b. the methods for culstering the behavioural traces and connecting them to the tasks. In general there are two approaches to describe tasks: a. model the tasks and goals in advance. This can be achieved by describing tasks hierarchically (\cite{newell1972human}), or as a sequence of actions with a defined (\cite{eder1995workflow}) order. If actions and tasks are not described in advance, they usually do not have a pre-defined order or structure. In this case, machine-learning technologies are used to extract regularities that can be named as tasks (\cite{schmitz2011contextualized}). The second approach is eligible as the modelling of tasks is usually a very tedious assignment and then well-defined description do not match working processes in the real world. If task or coherent sequences of actions are found and named, the next job is to cluster them according to so-called activity schemes, that match the higher level descriptions of intentions as typical tasks of knowledge workers: Anyalse, acquire, disseminate, search and communicate information. With this again, typical classification of knowledge workers' roles shall be made possible: Learners, linkers, networkers etc. can be identified (\cite{reinhardt2011knowledge}). The machine-learning approaches will be explained in more detail in the next chapter. 

As a whole, the efforts explained belong to the research field of \ac{KM} and as such have the goal, in accordance with Taylor, to foster human capital and make resources available for companies \mnote{Knowledge Management}. As described in \: "'To compete effectively, firms must leverage their existing knowledge and create new knowledge that favorably positions them in their chosen markets \dots. (\textit{Knowledge Management}) must be present in order to store, transform and transport knowledge throughout the organization"' (\cite{gold2001knowledge}). This happens, as according to Taylor, in a mutual agreement (\cite{taylor2013scientific}, p.10):
\begin{quotation}
Scientific management \dots has for its very foundation the firm conviction that the true interests of the two (\textit{employé and employer}) are the one and the same; that prosperity for the employer cannot exist through a long term of years unless it is accompanied by prosperity for the employé, and vice versa.
\end{quotation}

\section{Machine Learning and Knowledge Management}
Most KM and \ac{DM} techniques involve learning patterns from existing data or information, and are therefore built upon the foundation of machine learning and artificial intelligence. The primary techniques that can be used by the organizations usually are statistical analysis, pattern discovery and outcome prediction. A variety of non-typical data can be similarly monitored. Before the advent of \acs{DM} and \acs{KM} techniques, the organizations relied almost exclusively on human expertise (\cite{tsai2012knowledge}). In the following the general approaches in \acs{DM} for \acs{KM} are discussed.

\subsection{Supervised Machine Learning}
A typical example for supervised task learning is an approach called 'bag of words'-modeling. In the 'bag of words'-model a text is represented as an unordered collection of words, where the frequency of each word is used as a feature for training a classifier. \cite{granitzer2008analysis} use this basic method from the field of \ac{IR} as the input for their classifiers which is the final step of their processing pipeline. On the first level, the so-called \textit{data acquisition}, raw event data from the operating system is collected. These are keystrokes, mouse clicks and used applications as well as file names, file authors, document structure etc. User actions and operating system reactions are called \textit{events} (see fig. \ref{fig2}).  
\begin{figure}[ht]
	\centering
  \includegraphics[width=240px]{granitzer}
	\caption{Classifying events}
	\label{fig2}
\end{figure}

In this case, subsequent events are aggregated to so-called \textit{event blocks}. The rule for creating these blocks can be 'time': events that take place within a small time period, or semantic characteristics defined by applications like \textit{editing} a text file. An example for this mapping would be: A user opens a text file with his text processing application, navigates to a certain paragraph, begins reading and then writing (as reading is usually recognized by scrolling within the application). Events then are clustered according to certain \textit{features} that are extracted from the event blocks. The features used by the authors were: Application name, window title, content and semantic type. Of these the semantic type is the prevalent feature described above for building event blocks, if the application provides according detailed information. 



  
