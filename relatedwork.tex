% This file is part of the i10 thesis template developed and used by the
% Media Computing Group at RWTH Aachen University.
% The current version of this template can be obtained at
% <http://www.media.informatik.rwth-aachen.de/karrer.html>.

\chapter{Related work}
\label{relatedwork}
\section{Knowledge and Task Mining}
In the computer scientific field of knowledge and task mining is no new subject.\mnote{Context-Aware Systems} With the rise of mobile computing devices the term context-aware systems was created. The meaning and definition are disputed. First publications referred to a user's location: in different places usually different contextual parameters are relevant. For example a diver that is ascending from deep water has to be made aware of resting times before emerging to the surface.Another example was the \textit{Active Badge Location System} in 1992 that detected the whereabouts of a person and in order to forward relevant phone calls to telephones close targeted person (\cite{want1992active}). Such systems adapt not only to the location but also to other relevant and changing parameters in the surroundings (\cite{schilit1994context}). This definition was widened in 1998 where context was referred to not only the computer accessible parameters of the surroundings but also the emotional state, focus of attention, date and time as well as people in the environment (\cite{dey1998context}). The new aspect of internal parameters like focus of attention was then referred to a further elaboration of the definition: the internal (logical) and external context: Internal context parameters are specified by the user in interaction with the computer like goals, tasks, work context, business processes and emotional state. External parameters are usually measured by hardware sensors, i.e. location, light, sound, movement, temperature, pressure etc. (\cite{hofer2003context}). The contextual parameters can be grouped into four categories: identity (marked by a unique identifier), the location (an entity’s position), activity (status, meaning the intrinsic properties of an entity, e.g., temperature and lightning for a room, processes running currently on a device etc.) and time (timestamps, \cite{dey2001conceptual}). An example of the use of internal data for extracting context is the \textit{Watson Project} \cite{budzik2000user}. Here the focus is shiftet for collecting contextual information from user interaction with the computer in order to proactively support the user. Proactivity is a term that  originates in organizational psychology and describes the ability of workers to not react to situations, but sense upcoming situational changes in advance and take control (\cite{grant2008dynamics}). As work gets more dependent of the retrieval and analysis of information, a proactive support system shall help the user in his various tasks by providing him with relevant information. This approach had further implications as gathering information from the user interaction with his computer requires techniques from information retrieval and computer linguistics. In this case the documents a user works with are analyzed and keywords are stored as vectors or a bag of words. The relevant keywords shall help to narrow the topical context a user is working on. Keywords than help to start searches with relevant search terms and provide the user with the information he needs (\cite{budzik2000user}). \mnote{Context-based Recommender Systems} As a single user is often not able to find the needed information, his typical search patterns are compared with those of other users. In these cases as \textit{user model} is created, and his search terms are compared to those of other users' and the documents they found. If keywords are matching, documents of those others users are recommended (\cite{anand2007contextual}). This approach is called context-based recommendation and their related techniques like user-collaborative filtering are applied in search engines like Amazon \footnote{www.amazon.com}. \textit{Attention-aware systems} at last have different focus: The guiding principle of Attention-aware systems is that users have limited cognitive resources and are distracted easily \mnote{Attention-aware systems}. They suffer from an \textit{information overload} as they jump quickly from one resource to the next in the same and different workings tasks. Whilst it is beneficial to be able to change foci in certain situations, in others it is exhausting. Therefore systems capable of supporting and guiding user attention have to assess the current user focus, and calculate the cost/benefits of attention shifts (interruptions). As this explanation shows, Attention-aware systems have a foundation in cognitive psychology. 
