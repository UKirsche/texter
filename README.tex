% This file is part of the i10 thesis template developed and used by the
% Media Computing Group at RWTH Aachen University.
% The current version of this template can be obtained at
% <http://www.media.informatik.rwth-aachen.de/karrer.html>.

\documentclass[11pt,a4paper]{article}

\usepackage[applemac]{inputenc}
\usepackage[pdftex]{hyperref}
\usepackage{palatino}
\usepackage{color,graphicx}
\definecolor{blue}{rgb}{0,0,1}
\newcommand{\mySimpleURL}[2]%
{%
	\textcolor{blue}{%
		\href{#2}{#1}%
	}%
	\footnote{#2}
}


\begin{document}
\sloppy
\section*{ReadMe---Deutsch}
Nur ganz kurz die wichtigsten Punkte zur i10 Diplomarbeitsvorlage:
\begin{itemize}
	\item Lest euch zuerst die Dateien \emph{main.tex} und \emph{i10preamble.tex} (im Finder orange hinterlegt) durch. Die Kommentare sind eigentlich recht ausf\"uhrlich.
	\item Falls ihr Bilder einbinden wollt (und das wollt ihr ja alle), speichert sie als PDF im \emph{images}-Verzeichnis ab. Dann k�nnt ihr die Befehle \emph{myFigure}, \emph{myBigFigure} und \emph{myHugeFigure} aus \emph{i10preamble.tex} verwenden, um die Bilder einzubinden (n\"aheres dazu steht in \emph{i10preamble.tex}).
	\item Eine Diplomarbeit, welche mit dieser Vorlage erstellt wurde k\"onnt ihr euch unter \mySimpleURL{media.informatik.rwth-aachen.de/karrer.htm}{http://media.informatik.rwth-aachen.de/karrer.htm} zur Ansicht herunterladen.
	\item Die Titelseite liegt als fertiges PDF vor und kann mit Adobe Acrobat (ist auf den Lab-Rechnern am i10 installiert) bearbeitet werden.
	\item Fragen, Anregungen, W\"unsche oder Probleme bitte per mail an mich (Link s.o.). Ich versuche mich schnellstm\"oglich darum zu k\"ummern.
\end{itemize}
%\vfill
%\pagebreak
\section*{ReadMe---English}
A few important things to know when working with the i10 thesis template:
\begin{itemize}
	\item First thing to do is reading \emph{main.tex} and \emph{i10preamble.tex} (labeled orange in the Finder). The files are excessively commented.
	\item If you plan on using figures, save them as PDF and store them inside of the \emph{images}-folder. You will then be able to use the commands \emph{myFigure}, \emph{myBigFigure}, and \emph{myHugeFigure} from the \emph{i10preamble.tex} to include the figures (more on this topic can be found in \emph{i10preamble.tex}).
	\item You can download a diploma thesis that was typeset using this template at \mySimpleURL{media.informatik.rwth-aachen.de/karrer.htm}{http://media.informatik.rwth-aachen.de/karrer.htm} to get a notion about how your thesis will look like.
	\item The titlepage can be edited using Adobe Illustrator (installed on the lab machines @ i10).
	\item Feel free to contact me if there are any problems with the template. I will try to find a solution ASAP.
\end{itemize}


\end{document}